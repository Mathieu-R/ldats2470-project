\section*{Appendix}

\subsection*{Description of the different variables}

The \textbf{response variable} is \textit{status}: $1$ if the subject has the Parkinson disease and $0$ if not.

The \textbf{explanatory variables} are the following:

\begin{itemize}
	\item \textit{name}: the suject name along the recording number.
	\item \textit{mdvp.fo}: the \textbf{average} local fundamental frequency.
	\item \textit{mdvp.fhi}: the \textbf{maximum} local fundamental frequency.
	\item \textit{mdvp.flo}: the \textbf{minimum} local fundamental frequency.
	\item \textit{mdvp.jitter\_perc ($\%$)}, \textit{mdvp.jitter\_abs (Abs)}, \textit{mdvp.rap}, \textit{mdvp.ppq}, \textit{mdvp.ddp}: these are several measures of variation in fundamental frequency.
	\item \textit{mdvp.apq}, \textit{mdvp.shimmer}, \textit{mdvp.shimmer\_db}, \textit{shimmer.apq3}, \textit{shimmer.apq5},\textit{shimmer.dda}: these are several measures of variation in amplitude.
	\item \textit{nhr}, \textit{hnr}: 2 measures of noise to tonal components in the voice.
	\item \textit{rdpe}, \textit{d2}: 2 nonlinear dynamical complexity measures.
	\item \textit{dfa}: signal fractal scaling exponent.
	\item \textit{spread1,} \textit{spread2}, \textit{ppe}: 3 nonlinear measures of fundamental frequency variation.
\end{itemize}

